\documentclass[12pt,twoside]{report}
\usepackage{physics}
\usepackage{amsmath,amssymb,amsthm}
\usepackage{braket}
\usepackage[utf8]{inputenc}
\usepackage{showlabels}
\usepackage{cases}
\usepackage{mathtools}
\usepackage[a4paper,width=150mm,top=25mm,bottom=25mm,bindingoffset=6mm]{geometry}
\usepackage{xcolor}

\title{The  notice from exact ,apprximate ,and numerical calculation of GES}
\author{talipjan}

\begin{document}
	
			
	\paragraph{reference concepts}
	lemma6.Let there be given  a  set of product vectors $B={\ket{\varPsi _{x}}  \otimes \ket{\Psi_{x}} }_{x}$ from $ C^{m}\otimes C^{n} $ with cardinality $ |B|\geq m+n-1. $ if any m-tuple of vectors $ \ket{\Psi_{x}} $ span $ C^{m} $and  any n-tuple of vectors $ \ket{\varPsi_{x}} $ span $ C^{n} $,then there is no product vector in the orthocomplement of span B.	
	\newline\\
	while we care about linear indepence of the coordinates,at the same time we require the condition dim span B$ < d^{N} $ to hold ,that is ,the resulting GES to be nonempty.
\paragraph{our cnsideration}
   In our case we consider multiple qubit Hilbert spaces, i.e.,$ H_{2^{3}} :=(C^{2})^{\otimes3}$
   \\
   \begin{equation*}
B= \left\lbrace {\Ket{\Psi}|\ket{\Psi}=(1,a+b\alpha+c\alpha^{2})_{A}\otimes(1,A+B\alpha+C^{2})_{B}\otimes(1,x+y\alpha+Z\alpha^{2})_{C}}  \ |\alpha\in C\right\rbrace 
   \end{equation*}
   \\   
   \\
    where a,b,c,A,B,C,XY,Z$ \in \{-1,0,1\} $
   \\
   \\
   First of all,we  move main detail of discusstion,give some direct calculation on it.
   B=$ \left\lbrace {\Ket{\Psi}|\ket{\Psi}=(1,a+b\alpha+c\alpha^{2})_{A}\otimes(1,A+B\alpha+C^{2})_{B}\otimes(1,x+y\alpha+Z\alpha^{2})_{C}}  \ |\alpha\in C\right\rbrace $
  \\
  \\
   $A|BC$
  \\
  AB: $ (1,a+b\alpha+c\alpha^{2})_{A}\otimes(1,A+B\alpha+C^{2})_{B}\ =(1,a+b\alpha+c\alpha^{2},A+B\alpha+C^{2},Aa+(Ba+Ab+Bc)\alpha+(Ca+Bb+Ac+Cc)\alpha^{2}+(Cb+Bc)\alpha^{3}+Cc\alpha^{4})$ Assume it is coordinates with respect  $ \alpha   $ linearly independent polynomials have dimension is 4.  (depending on spanning property)
  \\
  C:$ (1,x+y\alpha+Z\alpha^{2}) $ linearly independt polynomial dimestion is 2.(depending on spanning property)
  \\
  \\
  $ B|AC $
  \\
  B:$ (1,A+B\alpha+C^{2}) $ dimmestion is 2
  \\
  AC:$ (1,a+b\alpha+c\alpha^{2})\otimes(1,x+y\alpha+Z\alpha^{2})=(1,a+b\alpha+c\alpha^{2},x+y\alpha+Z\alpha^{2},Xa+(Ya+Xb+Yc)\alpha+(Za+Yb+Xc+Zc)\alpha^{2}+(Zb+Yc)\alpha^{3}+Zc\alpha^{4}) $ dim is 4\\
  B:$ (1,A+B\alpha+C^{2}) $ dim is 2
  \\
  \\
  \\
  $ A|BC $\\
  \\
  BC:$ (1,A+B\alpha+C^{2})\otimes(1,x+y\alpha+Z\alpha^{2})=(1,A+B\alpha+C^{2},x+y\alpha+Z\alpha^{2},XA+(YA+XB+YC)\alpha+(ZA+YB+XC+ZC)\alpha^{2}+(ZB+YC)\alpha^{3}+Zc\alpha^{4}) $ dim is 4\\
  \\
  A:$(1,a+b\alpha+c\alpha^{2})$ dim is 2.
  \\  
  Second,let $\bar{B}$ be a subspace whose orthogonal to spane of B.and $\bar{B}$ span by $ \ket{\xi,\gamma}= \ket{\xi}_{S}\otimes\ket{\gamma}_{\bar{S}}=(\ket{\xi}_{0})_{S}\otimes(\Ket{\gamma}_{0},\ket{\gamma}_{1})_{\bar{S}}$
  \\
  we could easly say that ,the  tesnodr product  of the  coorosponding pair vecors in these two set are equal to zero.
 between two sets have bijective function relation.let say F:$ (C^{2})^{2} \longrightarrow (C^{2})^{2} $,or taking it a step further,we could say that F is a transformation.\\
 Definition: A linear transformation T : v$ \longrightarrow v $is said to be non-singular if T(v) = 0 $ \Rightarrow v = 0  i.e. N(T) = {0}  $ 
 \\
 Definition: A linear transformation T : V is said to be
 singular if  some $ \exists    v  \epsilon   V s.t. v\neq 0   \ \ and \ \ T(v) = 0 $ i.e. N(T) contains at least one-zero element. 
  \\
  Definition:A linear transformation is an isomorphism if it
  is one-one and onto.
  i.e. T : $V \longrightarrow W$is an isomorphism if\\
  (1) T is linear transformation.\\
  (2) T is one-one.\\
  (3) T is onto.\\
  \\
  Then V and W are called isomorphic.\\
  We write V $ \cong $ W\\
  \\
  THEOREM: V $ \cong $ W $ \Leftrightarrow $dim V = dim W
  cooroseponding proof in  silde in attachment.\\
  \\
  \\
  from the definition of linear transformation ,we could say that F:$ (C^{2})^{2} \longrightarrow (C^{2})^{2} $ could generate a Matrix F.and matrix F is consist of $ A,B,C $\\
  if we choose a $ S| \bar{S} $ bipartition,$ S\cup \bar{S} =A$ ,where local dimention of S and $ \bar{S} $ are 2,4 respectively.  
  the question will rise in here,which what about the order of bipartition cut ?
 \\
 the answer coming from Galois theory by Harlod .M.Edwards.page 47-55.content:Basic Galois Theory: The Galois Group (permutaion of the roots of polynomilas)
 \\
 for example we suppose S=$B  $ \ \ and$  \bar{S}=AC $ ,which mean we have a permutaion p(123)=213.it is easy to see that this order of permutaion is represent a unique individual bipartition. 
 \ \ \ \ \ \ \ \ \ our main operation on our case  staring from new.if we assume A biproduct vector  $ \ket{\xi,\gamma}= \ket{\xi}_{S}\otimes\ket{\gamma}_{\bar{S}}=(\ket{\xi}_{0})_{S}\otimes(\Ket{\gamma}_{0},\ket{\gamma}_{1})_{\bar{S}}$ belong to    $ \bar{B} $ ,it should satsify following eqaution:
 $ \bra{\zeta,\gamma}\ket{\Psi}=0 \ \  \forall \alpha $ where $ \ket{\Psi}\in B $\\
 \\
 $ A|BC cut $\\
 \\
 $ \bra{\xi}_{0}\ket{(1,a+b\alpha+c\alpha^{2})}\otimes\bra{\gamma_{0},\gamma}_{1}\ket{(1,A+B\alpha+C\alpha^{2}),(1,X+Y\alpha+Z\alpha^{2})}=0 $\\
 \\
 $ \bra{\zeta}_{0}\otimes(1,a+b\alpha+c\alpha^{2})\otimes\bra{\gamma_{0},\gamma}_{1}((1,A+B\alpha+C\alpha^{2}),(1,X+Y\alpha+Z\alpha^{2}),AX+(BaX+AY+BZ)\alpha+(AZ+YB+XC+ZC)\alpha^{2}+(ZB+YC)\alpha^{3}+ZC\alpha^{4})=0 $\\
 After sorting:\\
 $ \bra{\gamma_{0},\gamma}_{1}*(1+A+XA)+\bra{\gamma_{0},\gamma}_{1}*(B+YA+XB+YC)\alpha+(\bra{\gamma_{0},\gamma}_{1}*(Z+ZA+YB+XC+ZC)\alpha^{2}+\bra{\gamma_{0},\gamma}_{1}(ZB+YC)\alpha^{3}+\bra{\gamma_{0},\gamma}_{1}(ZC\alpha^{4}))=0$
 \\
 since left side of all above equations is a ploynomila degree 4 in variable $\alpha$.if we consider must be hold ,then the coefficients are qual to zero.which means :
 \\
 $ \bra{\gamma_{0},\gamma}_{1}*(1+A+XA) =0$\\
 $\bra{\gamma_{0},\gamma}_{1}*(B+YA+XB+YC)=0  $\\
 $ (\bra{\gamma_{0},\gamma}_{1}*(Z+ZA+YB+XC+ZC)=0 $\\
 $ \bra{\gamma_{0},\gamma}_{1}(ZB+YC)=0 $\\
 $ \bra{\gamma_{0},\gamma}_{1}(ZC)=0 $
 \\
 \\
 \begin{equation}
 E=\bra{\gamma_{1}}
 \begin{bmatrix}
 1+A+A+XA\\
 B+YA+XB+Yc\\
 Z+ZA+YB+XC+ZC\\
 ZB+YC\\
 ZC
 \end{bmatrix}
 \end{equation}
 \\
  if we assume $ \ket{\gamma_{0}}$ is a constant parameter,then every bipartition have 5 homogenous linear  equations with 2 unknowns .and we could write these linear homogenous equation in matrix form.\\
 $ B|AC cut $\\
 \\
 $ \bra{\xi}_{0}\ket{(1,A+B\alpha+C\alpha^{2})}\otimes\bra{\gamma_{0},\gamma}_{1}\ket{(1,a+b\alpha+c\alpha^{2}),(1,X+Y\alpha+Z\alpha^{2})}=0 $\\
 $ \bra{\zeta}_{0}\otimes(1,A+B\alpha+C\alpha^{2})\otimes\bra{\gamma_{0},\gamma}_{1}((1,a+b\alpha+c\alpha^{2}),(1,X+Y\alpha+Z\alpha^{2}),Xa+(Ya+Xb+Yc)\alpha+(Za+Yb+Xc+Zc)\alpha^{2}+(Zb+Yc)\alpha^{3}+Zc\alpha^{4})=0 $\\
 After sorting:\\
 $ \bra{\gamma_{0},\gamma}_{1}*(1+a+Xa)+\bra{\gamma_{0},\gamma}_{1}*(b+Ya+Xb+Yc)\alpha+(\bra{\gamma_{0},\gamma}_{1}*(Z+Za+Yb+Xc+Zc)\alpha^{2}+\bra{\gamma_{0},\gamma}_{1}(Zb+Yc)\alpha^{3}+\bra{\gamma_{0},\gamma}_{1}(Zc\alpha^{4}))=0$
 \\
 since left side of all above equations is a ploynomila degree 4 in variable $\alpha$.if we consider must be hold ,then the coefficients are qual to zero.which means :
 \\
 $ \bra{\gamma_{0},\gamma}_{1}*(1+a+Xa) =0$\\
 $\bra{\gamma_{0},\gamma}_{1}*(b+Ya+Xb+Yc)=0  $\\
 $ (\bra{\gamma_{0},\gamma}_{1}*(Z+Za+Yb+Xc+Zc)=0 $\\
 $ \bra{\gamma_{0},\gamma}_{1}(Zb+Yc)=0 $\\
 $ \bra{\gamma_{0},\gamma}_{1}(Zc)=0 $
 \\
 \begin{equation}
 E=\bra{\gamma_{1}}
 \begin{bmatrix}
 1+a+Xa\\
 b+Ya+Xb+Yc\\
 Z+Za+Yb+Xc+Zc\\
 Zb+Cc\\
 Zc
 \end{bmatrix}
 \end{equation}
 \\
  if we assume $ \ket{\gamma_{0}}$ is a constant parameter,then every bipartition have 5 homogenous linear  equations with 2 unknowns .and we could write these linear homogenous equation in matrix form.\\
  $ C|AB cut $\\
 \\
 $ \bra{\xi}_{0}\ket{(1,X+Y\alpha+Z\alpha^{2})}\otimes\bra{\gamma_{0},\gamma}_{1}\ket{(1,a+b\alpha+c\alpha^{2}),(1,A+B\alpha+C\alpha^{2})}=0 $\\
 $ \bra{\zeta}_{0}\otimes(1,X+Y\alpha+Z\alpha^{2})\otimes\bra{\gamma_{0},\gamma}_{1}((1,a+b\alpha+c\alpha^{2}),(1,A+B\alpha+C\alpha^{2}),Aa+(Ba+Ab+Bc)\alpha+(Ca+Bb+Ac+Cc)\alpha^{2}+(Cb+Bc)\alpha^{3}+Cc\alpha^{4})=0 $\\
 After sorting:\\
 $ \bra{\gamma_{0},\gamma}_{1}*(1+a+Aa)+\bra{\gamma_{0},\gamma}_{1}*(b+Ba+Ab+Bc)\alpha+(\bra{\gamma_{0},\gamma}_{1}*(C+Ca+BB+Ac+Cc)\alpha^{2}+\bra{\gamma_{0},\gamma}_{1}(Cbb+Bc)\alpha^{3}+\bra{\gamma_{0},\gamma}_{1}(Cc\alpha^{4}))=0$
 \\
 since left side of all above equations is a ploynomila degree 4 in variable $\alpha$.if we consider must be hold ,then the coefficients are qual to zero.which means :
 \\
 $ \bra{\gamma_{0},\gamma}_{1}*(1+a+Aa) =0$\\
 $\bra{\gamma_{0},\gamma}_{1}*(b+Ba+Ab+Bc)=0  $\\
 $ (\bra{\gamma_{0},\gamma}_{1}*(C+Ca+BB+Ac+Cc)=0 $\\
 $ \bra{\gamma_{0},\gamma}_{1}(Cbb+Bc)=0 $\\
 $ \bra{\gamma_{0},\gamma}_{1}(Cc)=0 $
 \\
 $  $
 \\
\begin{equation}
E=\bra{\gamma_{1}}
\begin{bmatrix}
b+Ba+Ab+Bc\\
C+Ca+Bb+Ac+Cc\\
Cb+Bc\\
Cc
\end{bmatrix}
\end{equation}
and this kind of matrix does not exit for satisfying $ \bra{\zeta,\gamma}\ket{\Psi}=0 \ \  \forall \alpha $ where $ \ket{\Psi}\in B $\\
 and let $ \bar{B} $ be a GES,it is requires that the system only has the trivial soultion.and it will happen when the matirx E of the sytem is full rank.
 r(E)=5,for all $ \bra{\gamma_{1}} $ is not same time equal to zero.
\end{document}