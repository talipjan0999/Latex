\documentclass[a4paper,12pt,landscape]{report}

	\usepackage{parallel}
\usepackage{kantlipsum}
\usepackage[margin=.5in]{geometry}


\usepackage{lipsum}
\usepackage{physics}
\usepackage{amsmath,amssymb,amsthm}
\usepackage{braket}
\usepackage[utf8]{inputenc}
\usepackage{showlabels}
\usepackage{cases}
\usepackage{mathtools}




	\begin{document}
		\begin{Parallel}[v]{0.49\textwidth}{0.49\textwidth}
			\ParallelLText{ PRELIMINARIES:Finite–dimensional product Hilbert spaces,
				denoted $H_{d_{1}},_{d_{2}}........,_{d_{N}}= C^{d_{1}} \otimes ....C^{d_{n}}$
				or $H_{d^{n}} = C^{d}\otimes C ^{d}.......\otimes C^{d}$.
				Subsystems
				are denoted $A_{1}, A_{2}, . . . , A_{n} $=A in the general multipartite case or A, B, forsmaller systems. For pure states we use the traditional denotations: $ \ket{\Psi} ,\ket{\Psi}$ , often
				adding subscripts corresponding to respective (groups of) parties, e.g., $ \ket{\Psi}_{ABC} $. We
				will use the standard basis for all the parties ${ \ket{i} }_{i=0} ^{d}$
				and the kets will be written as row vectors.\\
				Entanglement. An n–partite pure state$ \ket{\Psi}_{A_{1}A_{2}.........A_{n}} $ is said to be fully product if
				it can be written as  $ \ket{\Psi}_{A_{1}A_{2}.........A_{n}} =\ket{\Psi}_{A_{1}}\otimes\ket{\kappa}_{A_{2}}.......\otimes\ket{\vartheta}_{n}$ Otherwise it is called
				entangled. Among entangled states a particularly interesting class is constituted by
				genuinely multiparty entangled (GME) states, i.e., those which cannot be written
				as$ \ket{\Psi}_{A_{1}A_{2}.........A_{n}}=\ket{\Psi}_{S}\otimes\ket{\kappa}_{\bar{S}}$
				for any bipartite cut (biaprtition)$ S|\bar{S}$ where S is a
				subset of the parties and $ \bar{S}=A \backslash S $.In other words a GME state is not biproduct
				with respect to any bipartite cut of the parties. A canonical example of a GME state is
				the famous GHZ state$ \Ket{GHZ}=1/\surd2(\ket{000.....00}+\ket{111.......11}) $.
				A state $ \ket{\Psi} $ is called k– product if it is of the form $ \ket{\Psi_{\otimes^{k}}}=\ket{\Psi_{1}}_{s_{1}}\otimes \ket{\Psi_{2}}_{s_{2}}\otimes................ \otimes\ket{\Psi_{k}}_{s_{k}}$
				where $ S_{1}\cup S_{2}\cup S_{3}........\cup S_{K}=A $ is a k–partition. In the particular case k = n, the
				vector is fully product; when k = 2 it is biproduct.\\
				Completely and genuinely entangled subspaces. It is a  subspaces containing only entangled states, so called completely entagled
				subspaces  (CESs).  It has been shown that their maximal achievable dimension for $H_{d^{n}}$ is
				\begin{equation*}
				D_{max}^{CES}=.d^{n}+nd+n-1=(d^{n-1}+d^{n-2}+......+1-n)(d-1)
				\end{equation*} 
				\\ ]
			Definition 1. A subspace $\varrho \subset H_{d_{1},.........,d_{n}}$ is called a genuinely entangled subspace
			(GES) of\ \ $ H_{d_{1},.........,d_{n}} $ if any $ \ket{\Psi}\subset\varrho $ is genuinely multiparty entangled (GME).To obtain the maximal available dimension of a GES one needs to consider maximal
			dimensions of all bipartite GESs and take the smallest among them. It is then easy to
			see that for\\
			\begin{equation*}
			D_{max}^{GES}=(d^{n}-1)(d-1) 
			\end{equation*} 
			An example of a two dimensional GES of $ H_{2^{n}} $ is given by the span of} 
			\ParallelRText{ the already
				mentiond GHZ state and the W state,$ \ket{W}=\frac{1}{\surd n} (\ket{000.....01}+\ket{000.....010}+\ket{100.......00})$.
				\\
				\ \ \ \ \ \ we have given few other constructions of GESs
				working in general multiparty scenarios attaining larger dimensions. In particular, one
				of these constructions gives a GES of dimension $ d^{n-2}(d-1) $
			. Let us recall it here,
			for simplicity considering$ H_{3^{3}} $. Given is the set of vectors  $(a \in C):(1,\alpha+\alpha^{3},\alpha^{2}+\alpha^{6})\otimes(1,\alpha^{3},\alpha^{6})\otimes(1,\alpha,\alpha^{2}).$The subspace orthogonal to the span of these vectors
			is a twelve–dimensional GES. Choosing a set of twelve linearly independent vectors
			of the form above one obtains an example of a tripartite non–orthogonal unextendbile
			product basis.\\
			\paragraph{our consideration}
			In our case we consider multiple qubit Hilbert spaces, i.e.,$ H_{2^{3}} :=(C^{2})^{\otimes3}$
			\\
			\begin{equation*}
			B= \left\lbrace {\Ket{\Psi}|\ket{\Psi}=(1,a+b\alpha+c\alpha^{2})_{A}\otimes(1,A+B\alpha+C^{2})_{B}\otimes(1,x+y\alpha+Z\alpha^{2})_{C}}  \ |\alpha\in C\right\rbrace 
			\end{equation*}
			\\   
		In our cas, the condition that B is a GES is equivalent to saying that it is void of any
		biproduct vectors, i.e., we require vectors of the for $ \ket{\Psi}_{S}\otimes\ket{\kappa}_{\bar{S}} $, for any bipartition
		$S |\bar{S} $, not to belong to B. In other words, there can be no such vectors orthogonal to the
		subspace spanned by the vector in B (in any bipartate cut)\\
First of all,we  move main detail of discusstion,give some direct calculation 
on it.
	B=$ \left\lbrace {\Ket{\Psi}|\ket{\Psi}=(1,a+b\alpha+c\alpha^{2})_{A}\otimes(1,A+B\alpha+C^{2})_{B}\otimes(1,x+y\alpha+Z\alpha^{2})_{C}}  \ |\alpha\in C\right\rbrace $\\
	\\
		 $A|BC$
		\\
		AB: $ (1,a+b\alpha+c\alpha^{2})_{A}\otimes(1,A+B\alpha+C^{2})_{B}\ =(1,a+b\alpha+c\alpha^{2},A+B\alpha+C^{2},Aa+(Ba+Ab+Bc)\alpha+(Ca+Bb+Ac+Cc)\alpha^{2}+(Cb+Bc)\alpha^{3}+Cc\alpha^{4})$ Assume it is coordinates with respect  $ \alpha   $ linearly independent polynomials have dimension is 4.  (depending on spanning property)
		\\ C:$ (1,x+y\alpha+Z\alpha^{2}) $ linearly independt polynomial dimestion is 2.(depending on spanning property)} 
		\ParallelPar
		\end{Parallel}
	
	\begin{Parallel}[v]{0.49\textwidth}{0.49\textwidth}
		\ParallelLText{$ A|BC $\\
			BC:$ (1,A+B\alpha+C^{2})\otimes(1,x+y\alpha+Z\alpha^{2})=(1,A+B\alpha+C^{2},x+y\alpha+Z\alpha^{2},XA+(YA+XB+YC)\alpha+(ZA+YB+XC+ZC)\alpha^{2}+(ZB+YC)\alpha^{3}+Zc\alpha^{4}) $ dim is 4\\
			A:$(1,a+b\alpha+c\alpha^{2})$ dim is 2.
			\\  
			Second,let $\bar{B}$ be a subspace whose orthogonal to spane of B.and $\bar{B}$ span by $ \ket{\xi,\gamma}= \ket{\xi}_{S}\otimes\ket{\gamma}_{\bar{S}}=(\ket{\xi}_{0})_{S}\otimes(\Ket{\gamma}_{0},\ket{\gamma}_{1})_{\bar{S}}$
			\\
			we could easly say that ,the  tesnodr product  of the  coorosponding pair vecors in these two set are equal to zero.
			between two sets have bijective function relation.let say F:$ (C^{2})^{2} \longrightarrow (C^{2})^{2} $,or taking it a step further,we could say that F is a transformation.\\
			Definition: A linear transformation T : v$ \longrightarrow v $is said to be non-singular if T(v) = 0 $ \Rightarrow v = 0  i.e. N(T) = {0}  $ 
			\\
			Definition: A linear transformation T : V is said to be
			singular if  some $ \exists    v  \epsilon   V s.t. v\neq 0   \ \ and \ \ T(v) = 0 $ i.e. N(T) contains at least one-zero element. 
			\\
			Definition:A linear transformation is an isomorphism if it
			is one-one and onto.
			i.e. T : $V \longrightarrow W$is an isomorphism if\\
			(1) T is linear transformation.\\
			(2) T is one-one.\\
			(3) T is onto.\\
		
			Then V and W are called isomorphic.\\
			We write V $ \cong $ W\\
			THEOREM: V $ \cong $ W $ \Leftrightarrow $dim V = dim W
			cooroseponding proof in  silde in attachment.
			from the definition of linear transformation ,we could say that F:$ (C^{2})^{2} \longrightarrow (C^{2})^{2} $ could generate a Matrix F.and matrix F is consist of $ A,B,C $
			if we choose a $ S| \bar{S} $ bipartition,$ S\cup \bar{S} =A$ ,where local dimention of S and $ \bar{S} $ are 2,4 respectively.  
			the question will rise in here,which what about the order of bipartition cut ?
			\\
			the answer coming from Galois theory by Harlod .M.Edwards.page 47
		} 
		\ParallelRText{-55.content:Basic Galois Theory: The Galois Group (permutaion of the roots of polynomilas)
			\\
			for example we suppose S=$B  $ \ \ and$  \bar{S}=AC $ ,which mean we have a permutaion p(123)=213.it is easy to see that this order of permutaion is represent a unique individual bipartition. 
			\ \ \ \ \ \ \ \ \ our main operation on our case  staring from new.if we assume A biproduct vector  $ \ket{\xi,\gamma}= \ket{\xi}_{S}\otimes\ket{\gamma}_{\bar{S}}=(\ket{\xi}_{0})_{S}\otimes(\Ket{\gamma}_{0},\ket{\gamma}_{1})_{\bar{S}}$ belong to    $ \bar{B} $ ,it should satsify following eqaution:
			$ \bra{\zeta,\gamma}\ket{\Psi}=0 \ \  \forall \alpha $ where $ \ket{\Psi}\in B $\\
		 $ A|BC cut $\\
		\\
		$ \bra{\xi}_{0}\ket{(1,a+b\alpha+c\alpha^{2})}\otimes\bra{\gamma_{0},\gamma}_{1}\ket{(1,A+B\alpha+C\alpha^{2}),(1,X+Y\alpha+Z\alpha^{2})}=0 $\\
		\\
		$ \bra{\zeta}_{0}\otimes(1,a+b\alpha+c\alpha^{2})\otimes\bra{\gamma_{0},\gamma}_{1}((1,A+B\alpha+C\alpha^{2}),(1,X+Y\alpha+Z\alpha^{2}),AX+(BaX+AY+BZ)\alpha+(AZ+YB+XC+ZC)\alpha^{2}+(ZB+YC)\alpha^{3}+ZC\alpha^{4})=0 $\\
		After sorting:\\
		$ \bra{\gamma_{0},\gamma}_{1}*(1+A+XA)+\bra{\gamma_{0},\gamma}_{1}*(B+YA+XB+YC)\alpha+(\bra{\gamma_{0},\gamma}_{1}*(Z+ZA+YB+XC+ZC)\alpha^{2}+\bra{\gamma_{0},\gamma}_{1}(ZB+YC)\alpha^{3}+\bra{\gamma_{0},\gamma}_{1}(ZC\alpha^{4}))=0$
		\\
	since left side of all above equations is a ploynomila degree 4 in variable $\alpha$.if we consider must be hold ,then the coefficients are qual to zero.which means :
	\\
	$ \bra{\gamma_{0},\gamma}_{1}*(1+A+XA) =0$\\
	$\bra{\gamma_{0},\gamma}_{1}*(B+YA+XB+YC)=0  $\\
	$ (\bra{\gamma_{0},\gamma}_{1}*(Z+ZA+YB+XC+ZC)=0 $\\
	$ \bra{\gamma_{0},\gamma}_{1}(ZB+YC)=0 $\\
	$ \bra{\gamma_{0},\gamma}_{1}(ZC)=0 $
\begin{equation}
E=\bra{\gamma_{1}}
\begin{bmatrix}
1+A+A+XA\\
B+YA+XB+Yc\\
Z+ZA+YB+XC+ZC\\
ZB+YC\\
ZC
\end{bmatrix}
\end{equation}} 
		\ParallelPar
	\end{Parallel}
	
	
	\begin{Parallel}[v]{0.49\textwidth}{0.49\textwidth}
	\ParallelLText{if we assume $ \ket{\gamma_{0}}$ is a constant parameter,then every bipartition have 5 homogenous linear  equations with 2 unknowns .and we could write these linear homogenous equation in matrix form.\\
		$ B|AC cut $\\
		$ \bra{\xi}_{0}\ket{(1,A+B\alpha+C\alpha^{2})}\otimes\bra{\gamma_{0},\gamma}_{1}\ket{(1,a+b\alpha+c\alpha^{2}),(1,X+Y\alpha+Z\alpha^{2})}=0 $\\
		$ \bra{\zeta}_{0}\otimes(1,A+B\alpha+C\alpha^{2})\otimes\bra{\gamma_{0},\gamma}_{1}((1,a+b\alpha+c\alpha^{2}),(1,X+Y\alpha+Z\alpha^{2}),Xa+(Ya+Xb+Yc)\alpha+(Za+Yb+Xc+Zc)\alpha^{2}+(Zb+Yc)\alpha^{3}+Zc\alpha^{4})=0 $\\
		After sorting:\\
		$ \bra{\gamma_{0},\gamma}_{1}*(1+a+Xa)+\bra{\gamma_{0},\gamma}_{1}*(b+Ya+Xb+Yc)\alpha+(\bra{\gamma_{0},\gamma}_{1}*(Z+Za+Yb+Xc+Zc)\alpha^{2}+\bra{\gamma_{0},\gamma}_{1}(Zb+Yc)\alpha^{3}+\bra{\gamma_{0},\gamma}_{1}(Zc\alpha^{4}))=0$
		since left side of all above equations is a ploynomila degree 4 in variable $\alpha$.if we consider must be hold ,then the coefficients are qual to zero.which means :
		\\
	 $ \bra{\gamma_{0},\gamma}_{1}*(1+a+Xa) =0$\\
	$\bra{\gamma_{0},\gamma}_{1}*(b+Ya+Xb+Yc)=0  $\\
	$ (\bra{\gamma_{0},\gamma}_{1}*(Z+Za+Yb+Xc+Zc)=0 $\\
	$ \bra{\gamma_{0},\gamma}_{1}(Zb+Yc)=0 $\\
	$ \bra{\gamma_{0},\gamma}_{1}(Zc)=0 $
	\begin{equation}
	E=\bra{\gamma_{1}}
	\begin{bmatrix}
	1+a+Xa\\
	b+Ya+Xb+Yc\\
	Z+Za+Yb+Xc+Zc\\
	Zb+Cc\\
	Zc
	\end{bmatrix}
	\end{equation}
 if we assume $ \ket{\gamma_{0}}$ is a constant parameter,then every bipartition have 5 homogenous linear  equations with 2 unknowns .and we could write these linear homogenous equation in matrix form.\\
$ C|AB cut $\\
$ \bra{\xi}_{0}\ket{(1,X+Y\alpha+Z\alpha^{2})}\otimes\bra{\gamma_{0},\gamma}_{1}\ket{(1,a+b\alpha+c\alpha^{2}),(1,A+B\alpha+C\alpha^{2})}=0 $\\
$ \bra{\zeta}_{0}\otimes(1,X+Y\alpha+Z\alpha^{2})\otimes\bra{\gamma_{0},\gamma}_{1}((1,a+b\alpha+c\alpha^{2}),(1,A+B\alpha+C\alpha^{2}),Aa+(Ba+Ab+Bc)\alpha+(Ca+Bb+Ac+Cc)\alpha^{2}+(Cb+Bc)\alpha^{3}+Cc\alpha^{4})=0 $\\
After sorting:\\}
		\ParallelRText{$ \bra{\gamma_{0},\gamma}_{1}*(1+a+Aa)+\bra{\gamma_{0},\gamma}_{1}*(b+Ba+Ab+Bc)\alpha+(\bra{\gamma_{0},\gamma}_{1}*(C+Ca+BB+Ac+Cc)\alpha^{2}+\bra{\gamma_{0},\gamma}_{1}(Cbb+Bc)\alpha^{3}+\bra{\gamma_{0},\gamma}_{1}(Cc\alpha^{4}))=0$
			\\
			since left side of all above equations is a ploynomila degree 4 in variable $\alpha$.if we consider must be hold ,then the coefficients are qual to zero.which means :
			\\
			$ \bra{\gamma_{0},\gamma}_{1}*(1+a+Aa) =0$\\
			$\bra{\gamma_{0},\gamma}_{1}*(b+Ba+Ab+Bc)=0  $\\
			$ (\bra{\gamma_{0},\gamma}_{1}*(C+Ca+BB+Ac+Cc)=0 $\\
			$ \bra{\gamma_{0},\gamma}_{1}(Cbb+Bc)=0 $\\
			$ \bra{\gamma_{0},\gamma}_{1}(Cc)=0 $
			\\
			$  $
			\\
			if we assume $ \ket{\gamma_{0}}$ is a constant parameter,then every bipartition have 5 homogenous linear  equations with 2 unknowns .and we could write these linear homogenous equation in matrix form.\\
			\\
			\begin{equation}
			E=\bra{\gamma_{1}}
			\begin{bmatrix}
			b+Ba+Ab+Bc\\
			C+Ca+Bb+Ac+Cc\\
			Cb+Bc\\
			Cc
			\end{bmatrix}
			\end{equation}
		and this kind of matrix does not exit for satisfying $ \bra{\zeta,\gamma}\ket{\Psi}=0 \ \  \forall \alpha $ where $ \ket{\Psi}\in B $
		and let $ \bar{B} $ be a GES,it is requires that the system only has the trivial soultion.and it will happen when the matirx E of the sytem is full rank.
		r(E)=5,for all $ \bra{\gamma_{1}} $ is not same time equal to zero.	
this scenerio is same to in all bipartitaion cases.
from above ,it could give  an following theoroem\\
\textbf{Theorem:}Assume $\bar{B}$ is the subspace of $ H_{2^{3}}\subset (C^{2})^{n} $ orthogonal to 
spane of the vectors in B.then  $   \bar{B}  $ is a GES,and dimention is 3 ,satisfying this condition the matrix E for any bipartitation have  full rank with respect any   $ \gamma .$ 

}
			\ParallelPar
	\end{Parallel}
		
\begin{Parallel}[v]{0.49\textwidth}{0.49\textwidth}

			\ParallelLText{If we extend this situation to in general cases,then we have the following 
				theorem.\\
			\textbf{Theorem:}Assume $\bar{B}$ is the subspace of $ H_{2^{n}}\subset (C^{2})^{n} $ orthogonal to spane of the vectors in B.then  $   \bar{B}  $ is a GES,and dimention is $ 2^{n-1}-1 $ ,satisfying this condition the matrix E for any bipartitation have  full rank with respect any   $ \gamma .$
		
\textbf{EXAMPLE1.}Let the vectors spanning the subspace orthogonal toGES are given by following:\\
$A|BC$
\\
AB: $ (1,a+b\alpha+c\alpha^{2})_{A}\otimes(1,A+B\alpha+C\alpha^{2})_{B}\ =(1,a+b\alpha+c\alpha^{2},A+B\alpha+C^{2},Aa+(Ba+Ab+Bc)\alpha+(Ca+Bb+Ac+Cc)\alpha^{2}+(Cb+Bc)\alpha^{3}+Cc\alpha^{4})$  \\
,we could assign values for cofficients and  rewrite it in the form :\\
\begin{equation*}
(1,\alpha) _{A}\otimes (1,\alpha)_{B}(1,\alpha)_{C}  \ \  \forall \alpha\in C
\end{equation*}


\begin{equation*}
(1,\alpha) _{A}\otimes (1,\alpha,\alpha,\alpha^{2})_{BC}  \ \  \forall \alpha\in C   \\
\end{equation*}
BC spane 3 dimentional vector spave,\\
\begin{equation*}
(1,\alpha,\alpha,\alpha^{2},\alpha,\alpha^{2},\alpha^{2},\alpha^{3})_{ABC}
\end{equation*}
set of 4 linealrly independent polynomilays.which implies:
\begin{equation*}
u=dimspanB
\\
\end{equation*}
obiviusly,it is possible to chose u values of $\alpha $ so that the set 
\begin{equation*}
\bar{B}=\{\ket{\Psi(\alpha_{i})}\}_{i=1}^{4}   \ \      \forall \alpha \in C
\end{equation*}
fix vectors of $ \alpha $
\\
\begin{equation*}
(1,\alpha_{1})_{A}\otimes(1,\alpha_{1})_{B}\otimes(1,\alpha_{1})_{C} 
\end{equation*}

\begin{equation*}
(1,\alpha_{2})_{A}\otimes(1,\alpha_{2})_{B}\otimes(1,\alpha_{2})_{C}\\
\end{equation*}

\begin{equation*}
(1,\alpha_{3})_{A}\otimes(1,\alpha_{3})_{B}\otimes(1,\alpha_{3})_{C}\\
\end{equation*}

\begin{equation*}
(1,\alpha_{4})_{A}\otimes(1,\alpha_{4})_{B}\otimes(1,\alpha_{4})_{C}
\end{equation*}
\\
we need to check is there a vector in any bipartation  orthogonal to all the\ \ 
\\

}
                	\ParallelRText{ vectors above.\\
                		$ B|AC  $  bipartition \\
                		AC:$ (1,\alpha,\alpha,\alpha^{2}) $  spane 3 dimenstion vector space\\
                		$ (1,\alpha,\alpha,\alpha^{2},\alpha,\alpha^{2},\alpha^{2},\alpha^{3})_{ABC} $
                		set of 4 linearly independent polynomilas.
                		\begin{equation*}
(1,0)_{A}\otimes(1,0)_{B}\otimes(1,0)_{C} =(1,\alpha)_{A}\otimes(1,0,0,1)_{BC} 
                		\end{equation*}
                		
                		\begin{equation*}
(1,1)_{A}\otimes(1,1)_{A}\otimes(1,1)_{C} =(1,1)_{B}\otimes(1,1,1,1)_{BC}
                		\end{equation*}      
                		
                		\begin{equation*}
(1,-1)_{A}\otimes(1,-1)_{A}\otimes(1,-1)_{C}=(1,-1)_{B}\otimes(1,-1,-1,1)_{BC}  
                		\end{equation*}  	
                		
                		\begin{equation*}
(1,2)_{A}\otimes(1,2)_{A}\otimes(1,2)_{C}=(1,2)_{B}\otimes(1,2,2,4)_{BC}
                		\end{equation*}
  It is not hurd to see that the vector $ (2,-1)_{B}\otimes(0,1,-1,0)_{AC} $    orthogonal to all above vectors.          		
                		\\
                		\\
                		when we give some specific value for $ \alpha $,and the dimention of otrhocomplement of span$ \bar{B} $ are diffrent value.and the diffrent possiblit of value for cofficients in cordinates is  $3^{9} $
                	}
				\ParallelPar
		\end{Parallel}

\end{document}