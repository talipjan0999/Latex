\documentclass[a4paper,12pt]{article}
\usepackage{physics}
\usepackage{amsmath,amssymb,amsthm}
\usepackage{braket}
\usepackage[utf8]{inputenc}
\usepackage{showlabels}
\usepackage{cases}
\usepackage{mathtools}

\title{The short presentation about UPB ,UBB,CES,GES}
\author{talipjan}

\begin{document}
\leavevmode
\newline \\
A N-partite pure  state is in the form :
$ \ket{\Psi} _{A_{1}........A_{N}}\neq\ket{\Psi}_{A_{1}}\otimes......\Ket{\zeta}_{A_{N}}$ is called entaglement state.where the subindex represent of the local rank of subspace.
\  \  if it could write in the following form :
A N-partite pure  state is in the form :
$ \ket{\Psi} _{A_{1}........A_{N}}\neq\ket{\Psi}_{S}\otimes \Ket{\zeta}_{S^{-}}$ is called genuniely multiparty entagled state(GME).


on the other hand,A N-partite pure  state is in the form :
$ \ket{\Psi} _{A_{1}........A_{N}}=\ket{\Psi}_{S}\otimes\Ket{\zeta}_{S^{-}}$ is called biproduct state.

if we use this defination to mixed state:
\begin{equation}
\rho_{A}= \sum_{S|S_{-}}^{} [p_{S|S_{-}}] \sum_{i}^{} [q^{i}_{S|S_{-}}\varrho^{i}_{s}\otimes\sigma^{i}_{s}]
\end{equation}
 $ $
\end{document}