\documentclass[a4paper,12pt]{article}
\usepackage{physics}
\usepackage{amsmath,amssymb,amsthm}
\usepackage{braket}
\usepackage[utf8]{inputenc}
\usepackage{showlabels}
\usepackage{cases}
\usepackage{mathtools}

\begin{document}
Greenberger–Horne–Zeilinger state
The GHZ state is an entangled quantum state of$ M > 2$ subsystems. If each system has dimension ${\displaystyle d}$, i.e., the local Hilbert space is isomorphic to$ {\displaystyle \mathbb {C} ^{d}},$ then the total Hilbert space of M partite system is 
${\displaystyle {\mathcal {H}}_{tot}=(\mathbb {C} ^{d})^{\otimes M}}{\displaystyle {\mathcal {H}}_{tot}=(\mathbb {C} ^{d})^{\otimes M}}. $This GHZ state is also named asM-partite qubit GHZ state, it reads\\
$ \ket{GHZ}=\frac{1}{\surd d}\sum_{i=0}^{d-1} \ket{0}\otimes\ket{0}\otimes........\ket{0}+\ket{d-1} \otimes\ket{d-1}\otimes........\otimes\ket{d-1}$
\\
In the case of each of the subsystems being two-dimensional, that is for qubits, it reads$ \ket{GHZ}=\dfrac{\ket{0}^{M}+\ket{1}^{M}}{2} $\\
In simple words, it is a quantum superposition of all subsystems being in state 0 with all of them being in state 1 (states 0 and 1 of a single subsystem are fully distinguishable). The GHZ state is a maximally entangled quantum state.

The simplest one is the 3-qubit GHZ state:$ \ket{GHZ}=\dfrac{\ket{000}^{M}+\ket{111}^{M}}{3} $\
\\
Gram–Schmidt procedure\\
\begin{figure}[h!]
\includegraphics{../Untitled}
\caption{}
\label{fig:untitled}
\end{figure}
\end{document}